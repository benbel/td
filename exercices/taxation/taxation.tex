\documentclass[../main.tex]{subfiles}

\begin{document}

    \section{}

    \section{Exercices}

    \textbf{3.} Quel est le lien entre perte sèche et taux d'imposition ?
        \begin{enumerate}
            \item La perte sèche augmente plus que proportionnellement au taux d'imposition
            \item La perte sèche augmente proportionnellement au taux d'imposition
            \item La perte sèche augmente moins que proportionnellement au taux d'imposition
            \item La perte sèche augmente est indépendante du taux d'imposition
	    \end{enumerate}


    \textbf{1.} Si l'offre est parfaitement inélastique, quel serait l'incidence
        économique d'une taxe formellement payée par les producteurs ?
        \begin{enumerate}
            \item Le poids de la taxe serait uniquement supporté par les consommateurs
            \item Le poids de la taxe serait uniquement supporté par les producteurs
            \item Le poids de la taxe serait partagé entre producteurs et consommateurs
            \item L'information est insuffisante pour répondre
	    \end{enumerate}

    \textbf{3.} Si l'élasticité de l'offre de travail d'un individu au taux net
        de taxe est de 0.5 et que ce taux net de taxe passe de 50\% à 75\% \ldots
        \begin{enumerate}
            \item L'offre de travail va augmenter de 25 \% 
            \item L'offre de travail va augmenter de 12.5 \%
            \item L'offre de travail va augmenter de 50 \%
            \item L'information est insuffisante pour répondre
	    \end{enumerate}
    \textbf{5.} Si le marché du travail était en concurrence pure et parfaite, les cotisations
        sociales salariales et patronales :
        \begin{enumerate}
            \item Auraient la même incidence économique et la même incidence statutaire
            \item Auraient la même incidence économique et une incidence statutaire différente
            \item Auraient une incidence économique différente et la même incidence statutaire
            \item Auraient une incidence économique différente et lune incidence statutaire différente
	    \end{enumerate}


    \textbf{C.} Microsimulation \textit{4 points}

    Les questions suivantes portent sur le jeux de données \texttt{fisc.csv}
    joint aux exercices.

    Vous pouvez répondre aux questions suivantes avec un logiciel de type Excel
    ou autre.

    Le jeux données comporte des informations sur mille ménages (que l'on
    suppose constitués d'un homme, d'une femme et d'enfants par simplicité),
    numérotés de 1 à 1000, dont observe trois caractéristiques :
    \begin{itemize}
        \item le revenu du mari, \(z_{h}\)
        \item le revenu de la femme, \(z_{f}\)
        \item le nombre d'enfants, \(n_{e}\)
    \end{itemize}

    On étudie le système socio-fiscal suivant :
    \begin{itemize}
        \item l'impôt sur le revenu est un impôt par tranches :
            \begin{itemize}
                \item le taux marginal d'imposition est de 0\% entre 0 et 14 999
                    euros
                \item le taux marginal d'imposition est de 25\% entre 15 000 et
                    29 999 euros
                \item le taux marginal d'imposition est de 40\% au delà de 30
                    000 euros
            \end{itemize}
        \item le calcul de l'impôt se fait sur le quotient familial. Pour un
            ménage comptant \(n\) parts, le montant d'impôt dû est de \(n \times
            T\left(\frac{z_{h}+z_{f}}{n}\right)\)
        \item chaque individu du ménage (homme, femme ou enfant) constitue une
            part fiscale
    \end{itemize}

    \textbf{1.} Quel est le revenu brut médian dans cet échantillon ? Quel est le
    revenu brut moyen ?

    \textbf{2.} Calculez pour chacun des ménages le montant de leur impôt.
    Quel est le revenu net médian ? Quel est le revenu moyen médian ?
    Quel est le taux d'imposition effectif moyen ?

    \textbf{3.} Représentez le taux d'imposition moyen en fonction du revenu du
    ménage.

    On s'intéresse à la réforme fiscale suivante :
    \begin{itemize}
        \item chaque ménage reçoit une aide (non taxée) de 1000 euros par enfant
        \item le taux marginal de la dernière tranche passe de 30\% à 31\%
    \end{itemize}

    \textbf{4.} Quel est le coût pour les finances publiques de cette aide ?
    Quel est le gain pour les finances publiques de l'augmentation du taux marginal de la dernière tranche ?

    \textbf{5.} Cette réforme réduit-elle les inégalités ?

    Les questions suivantes portent sur le jeux de données \texttt{fisc.csv}
    joint aux exercices.

    Vous pouvez répondre aux questions suivantes avec un logiciel de type Excel
    ou autre.

    Le jeux données comporte des informations sur mille ménages (que l'on
    suppose constitués d'un homme, d'une femme et d'enfants par simplicité),
    numérotés de 1 à 1000, dont observe trois caractéristiques :
    \begin{itemize}
        \item le revenu du mari, \(z_{h}\)
        \item le revenu de la femme, \(z_{f}\)
        \item le nombre d'enfants, \(n_{e}\)
    \end{itemize}

    On étudie le système socio-fiscal suivant :
    \begin{itemize}
        \item l'impôt sur le revenu est un impôt par tranches :
            \begin{itemize}
                \item le taux marginal d'imposition est de 0\% entre 0 et 14 999
                    euros
                \item le taux marginal d'imposition est de 25\% entre 15 000 et
                    29 999 euros
                \item le taux marginal d'imposition est de 40\% au delà de 30
                    000 euros
            \end{itemize}
        \item le calcul de l'impôt se fait sur le quotient familial. Pour un
            ménage comptant \(n\) parts, le montant d'impôt dû est de \(n \times
            T\left(\frac{z_{h}+z_{f}}{n}\right)\)
        \item chaque individu du ménage (homme, femme ou enfant) constitue une
            part fiscale
    \end{itemize}

    \textbf{1.} Quel est le revenu brut médian dans cet échantillon ? Quel est le
    revenu brut moyen ?

    \dotfill

    Le revenu brut moyen est de 27203 euros et le revenu brut médian est de 21018 euros.

    \dotfill

    \textbf{2.} Calculez pour chacun des ménages le montant de leur impôt.
    Quel est le revenu net médian ? Quel est le revenu moyen médian ?
    Quel est le taux d'imposition effectif moyen ?

    \dotfill

    Le revenu net moyen est de 26216 euros et le revenu net médian est de 21018 euros.

    Si l'on calcule la somme des impôt payés (987257.4 euros) et la somme des revenus (27202977 euros),
    on obtient un taux moyen de 3.63\%.

    \dotfill

    \textbf{3.} Représentez le taux d'imposition moyen en fonction du revenu du
    ménage.

    \dotfill

    \begin{figure}[p]
        \centering
        \includegraphics[width=16cm]{taux_revenu}
        \caption{Taux effectif d'imposition en fonction du revenu du ménage}
    \end{figure}

    \dotfill

    On s'intéresse à la réforme fiscale suivante :
    \begin{itemize}
        \item chaque ménage reçoit une aide (non taxée) de 1000 euros par enfant
        \item le taux marginal de la dernière tranche passe de 30\% à 31\%
    \end{itemize}

    \textbf{4.} Quel est le coût pour les finances publiques de cette aide ?
    Quel est le gain pour les finances publiques de l'augmentation du taux marginal de la dernière tranche ?

    \dotfill

    Il y a 1 987 enfants dans la population, le coût de l'aide est donc de 1 987 000 euros.

    Le montant d'impôt collecté avant réforme est de 987257.4 euros ; après réforme il est de 
    1 072 055 euros : celle-ci rapporte donc 84797.6 euros.

    \dotfill

    \textbf{5.} Cette réforme réduit-elle les inégalités ?

    \dotfill

    Oui, il est d'ailleurs possible de le montrer sans calcul :
    \begin{itemize}
        \item Le nombre d'enfants est indépendant du revenu : l'aide n'a donc aucun effet redistributif vertical en moyenne
        \item L'augmentation du taux de la dernière tranche réduit le revenu des plus riches sans modifier celui-des autres, et réduit
              donc l'écart entre les plus riches et les plus pauvres.
    \end{itemize}

    Parmi les indicateurs d'inégalités, usuellement utilisés, on peut mentionner :
    \begin{itemize}
        \item Le rapport interdécile \textsc{d}9 / \textsc{d}1, c'est à dire le rapport du revenu du neuvième décile sur celui du premier décile.
        \item Le coefficient de Gini
        \item La part des différents quantiles dans le total des revenus
    \end{itemize}

    \section*{Séance 3}

    \textbf{A.} \textit{5 points}

    On étudie le comportement d'un individu \(i\), dont l'utilité dépend de
    sa consommation \( C_{i} \) et du nombre d'heures travaillées \( h_{i} \) :

    \[ U\left(C_{i}, h_{i} \right) = C_{i} - \frac{h_{i}^{2}}{2} \]

    Chaque heure travaillée est rémunérée à un salaire horaire \( s_{i} \).
    Le revenu total de l'individu est taxé à un taux \(\tau\).

    \textbf{1.} Exprimez la contrainte de budget de l'individu en fonction de
    \( s_{i}, h_{i}, \tau \) et \(C_{i}\).

    \textbf{2.} Supposez que l'individu maximise son utilité. Quel nombre d'heures
    \( h_{i}^{*} \) choisit-il de travailler, en fonction de \(s_{i}\) et \(\tau\)
    ? Quel est alors son revenu avant impôt ? Quel est son revenu après impôt ?

    \textbf{3.} Rappelez la définition d'une élasticité. Quelle est l'élasticité de
    l'offre de travail de l'agent (c'est à dire \( h_{i}^{*} \)) par rapport à son
    taux net de taxe (c'est à dire \( 1 - \tau\)) ?

    \textbf{4.} Rappelez ce que représente la courbe de Laffer. Pourquoi son maximum
    n'est-il pas atteint quand \( \tau = 1 \) ?

    \textbf{5.} La taxation présentée dans cet exercice est-elle progressive ?


    \textbf{B.} \textit{3 points}

    L'impôt sur le revenu est en France payé par les ménages fiscaux ; une
    alternative possible serait que les deux adultes du ménage payent séparement
    l'impôt sur leur revenu.

    En notant \( T\left(z\right) \) les impôts payés pour un revenu \( z \) :
    \begin{itemize}
        \item Dans le premier cas, on applique le barème de l'impôt sur le revenu
              sur la somme des revenus du ménage divisée par le nombre de parts
              fiscales, et l'on multiplie le résultat par le nombre de parts fiscales.
              Autrement dit, l'impôt payé est, pour un couple :
              \[ T\left(\frac{z_{h} + z_{f}}{2}\right) \times 2 \]

        \item Dans le second cas, on applique le barème de l'impôt sur le revenu
              à chacun des membres du ménage.
              Autrement dit, l'impôt payé est, pour un couple :
              \[ T\left(z_{h}\right) + T\left(z_{f}\right) \]
    \end{itemize}

    \textbf{1.} Sachant que l'impôt sur le revenu est progressif, expliquez pourquoi
        la taxation du ménage (premier cas) plutôt que des individus (deuxième cas)
        avantage les couples dont les revenus sont inégalitaires.

    Supposez que, comme dans la majorité des cas, le revenu \( z_{f} \) de la femme soit
    inférieur au revenu \( z_{h} \) du mari.

    \textbf{2.} Quel est le taux marginal d'imposition du revenu des
    femmes dans le premier et dans le second cas ?
    Comment l'offre de travail des femmes est-elle affectée ?


    Les questions suivantes portent sur l'article \og Are housing benefit an
    effective way to redistribute income? Evidence from a natural experiment
    in France \fg{}, de Gabrielle Fack, paru dans Labour Economics en 2006.

    \textbf{1.} Rappelez ce que sont l'incidence statutaire et l'incidence
    économique d'une taxe ou d'un transfert. Pourquoi diffèrent-elles ?

    \textbf{2.} Comment l'autrice mesure-t'elle l'incidence des \textsc{apl} ?
    Quelle variation exploite-t'elle ? Quels sont les groupes de contrôle et
    de traitement ? La méthode employée vous paraît-t'elle pertinente ?

    \textbf{3.} Quelle est l'incidence estimée ? Qu'en concluez-vous sur
    l'efficacité des \textsc{apl} pour réduire le montant du loyer des
    bénéficiaires ?


    \section*{Séance 3}

    \textbf{A.} \textit{5 points}

    On étudie le comportement d'un individu \(i\), dont l'utilité dépend de
    sa consommation \( C_{i} \) et du nombre d'heures travaillées \( h_{i} \) :

    \[ U\left(C_{i}, h_{i} \right) = C_{i} - \frac{h_{i}^{2}}{2} \]

    Chaque heure travaillée est rémunérée à un salaire horaire \( s_{i} \).
    Le revenu total de l'individu est taxé à un taux \(\tau\).

    \textbf{1.} Exprimez la contrainte de budget de l'individu en fonction de
    \( s_{i}, h_{i}, \tau \) et \(C_{i}\).

    \textbf{2.} Supposez que l'individu maximise son utilité. Quel nombre d'heures
    \( h_{i}^{*} \) choisit-il de travailler, en fonction de \(s_{i}\) et \(\tau\)
    ? Quel est alors son revenu avant impôt ? Quel est son revenu après impôt ?

    \textbf{3.} Rappelez la définition d'une élasticité. Quelle est l'élasticité de
    l'offre de travail de l'agent (c'est à dire \( h_{i}^{*} \)) par rapport à son
    taux net de taxe (c'est à dire \( 1 - \tau\)) ?

    \textbf{4.} Rappelez ce que représente la courbe de Laffer. Pourquoi son maximum
    n'est-il pas atteint quand \( \tau = 1 \) ?

    \textbf{5.} La taxation présentée dans cet exercice est-elle progressive ?

    \dotfill

    La contrainte de budget est :
    \[ \left( 1 - \tau \right) hs = C\]

    On peut alors substituer cette expression dans la fonction d'utilité :
    \[ U =  \left( 1 - \tau \right) hs - \frac{h^{2}}{2} \]

    Le maximum est atteint lorsque la dérivée de \(U\) par rapport aux variables
    de choix est nulle :
    \[ \frac{dU}{dh} = \left( 1 - \tau \right) s - h \]

    On en déduit que \(h^{*} = \left(1 - \tau \right) s\).

    Le revenu brut (\og avant\fg{} impôt) est de \(h^{*}s = \left( 1 -
    \tau\right) s \times s = \left( 1 - \tau \right) s^{2} \) et le revenu net
    (\og après \fg{} impôt) est de \(\left(1 - \tau\right) h^{*}s = \left(1 -
    \tau\right)^{2} s^{2}\).

    L'élasticité est la variation relative (en pourcentage) d'une variable par
    rapport à la variation relative d'une autre variable. Autrement dit
    l'élasticité \(\epsilon\) de \(Y\) par rapport à \(X\) est telle que si
    \(X\) augmente de \(1 \%\), \(Y\) varie de \(\epsilon \%\).

    L'élasticité de l'offre de travail par rapport au taux net de taxe est :
    \begin{align*}
        \epsilon &= \frac{1 - \tau}{h^{*}} \frac{\partial h^{*}}{\partial 1 -
        \tau} \\
                 &= \frac{1 - \tau}{\left( 1 - \tau \right) s} \frac{\partial
                 \left(1 - \tau\right) s}{\partial 1 - \tau} \\
                 &= \frac{1}{s} s \\
                 &= 1
    \end{align*}

    Lorsque le taux net de taxe augmente d'1\% (par exemple, si il passe de 0.6
    à 0.606), le nombre d'heures travaillées augmenterait de 1\% également.

    L'impôt collecté est de \( \tau \times R = \tau \times \left(1 - \tau\right)
    s^{2} \).

    Le taux d'imposition a deux effets de sens opposé sur le montant de l'impôt
    collecté :
    \begin{itemize}
        \item Il augmente la part (\(\tau\)) de l'assiette (c'est à dire \(\left( 1 - \tau \right) s^{2}\)) qui est collectée
        \item Il réduit la taille de l'assiette (c'est à dire de \(\left( 1 - \tau \right) s^{2}\))
              et donc, à taux égal, le montant collecté
    \end{itemize}

    Généralement, le premier effet domine lorsque \(\tau\) est faible et le
    second domine lorsque \(\tau\) est élevé.

    Si l'on représente le montant d'impôt collecté en fonction du taux
    d'imposition, graphique que l'on nomme \og courbe de Laffer \fg{}, la courbe
    est d'abord croissante, puis décroissante.

    Dans notre cas, le taux maximal est atteint pour \(\tau = 0.5\) (vous pouvez
    faire le calcul en dérivant le montant collecté par rapport à \(\tau\) pour
    trouver le \(\tau\) qui maximise le montant d'impôt collecté).

    L'impôt ici étudié est proportionnel au revenu, et donc ni régressif, ni
    progressif.

    \dotfill

    \textbf{B.} \textit{3 points}

    L'impôt sur le revenu est en France payé par les ménages fiscaux ; une
    alternative possible serait que les deux adultes du ménage payent séparement
    l'impôt sur leur revenu.

    En notant \( T\left(z\right) \) les impôts payés pour un revenu \( z \) :
    \begin{itemize}
        \item Dans le premier cas, on applique le barème de l'impôt sur le revenu
              sur la somme des revenus du ménage divisée par le nombre de parts
              fiscales, et l'on multiplie le résultat par le nombre de parts fiscales.
              Autrement dit, l'impôt payé est, pour un couple :
              \[ T\left(\frac{z_{h} + z_{f}}{2}\right) \times 2 \]

        \item Dans le second cas, on applique le barème de l'impôt sur le revenu
              à chacun des membres du ménage.
              Autrement dit, l'impôt payé est, pour un couple :
              \[ T\left(z_{h}\right) + T\left(z_{f}\right) \]
    \end{itemize}

    \textbf{1.} Sachant que l'impôt sur le revenu est progressif, expliquez pourquoi
        la taxation du ménage (premier cas) plutôt que des individus (deuxième cas)
        avantage les couples dont les revenus sont inégalitaires.

    Supposez que, comme dans la majorité des cas, le revenu \( z_{f} \) de la femme soit
    inférieur au revenu \( z_{h} \) du mari.

    \textbf{2.} Quel est le taux marginal d'imposition du revenu des
    femmes dans le premier et dans le second cas ?
    Comment l'offre de travail des femmes est-elle affectée ?

    \dotfill

    L'intuition est la suivante.

    L'imposition commune conduit à imposer les deux
    membres du foyer à un même taux intermédiaire entre les taux individuels.

    L'impôt étant progressif, le taux d'imposition moyen de l'individu
    le moins bien rémunéré est inférieur au taux intermédiaire, lui-même
    inférieur au taux moyen de l'individu le mieux rémunéré.

    Pour l'individu le moins bien rémunéré, cela aboutit à une hausse
    de son taux moyen et donc de l'impôt payé. Pour l'individu le mieux payé,
    cela aboutit à une baisse de son taux moyen et donc de l'impôt payé.

    Le second effet domine le premier car la part des revenus de l'individu le
    mieux payé est par définition la plus importante.

    Autrement dit, passer à l'imposition commune augmente les impôts sur une
    faible part des revenus (ceux de la personne la moins bien rémunérée) et les
    diminue sur une grande part des revenus (ceux de la personne la mieux
    rémunérée).

    Pour s'en convaincre, étudions le cas suivant :
    \begin{itemize}
        \item Un couple égalitaire où \(z_{h} = z_{f} = 20 000\)
        \item Un couple inégalitaire où \( z_{h} = 40 000\) et \(z_{f} = 0\)
        \item Un système d'impôt simple où \(T\left(z\right) =
            \left(\frac{z}{1000}\right)^{2}\)
    \end{itemize}

    Pour le couple égalitaire, il paye \(T\left(20 000\right) +
    T\left(20000\right) = 400 + 400 = 800\) euros d'impôt avec l'imposition
    séparée et \(2 \times T\left( \frac{20000 + 20000}{2}\right) = 2 \times
    T\left(20000\right) = 2 \times 400 = 800\) euros avec l'imposition
    jointe.

    Pour le couple inégalitaire, la femme paye \(T\left(0\right) = 0\) euros d'impôt
    (soit un taux moyen de \(0\)) et le mari paye \(T\left(40000\right) = 1600\)
    euros d'impôt (soit un taux moyen de 0.04).

    Dans le cas de l'imposition jointe, le foyer paye un impôt de \(2 \times
    T\left( \frac{0 + 40000}{2}\right) = 2 \times T\left( 20000\right) = 2
    \times 400 = 800\) euros d'impôt, soit un taux moyen de 0.02.

    L'imposition a deux conséquences :
    \begin{itemize}
        \item Le revenu de la femme est taxé à hauteur de 0.02 au lieu de l'être
            avec un taux de 0, ce qui n'a pas d'effet puisque \(z_{f} = 0\)
        \item Le revenu du mari est taxé à hauteur de 0.02 au lieu de l'être
            avec un taux de 0.04, ce qui divise par deux le montant d'impôt
            payé.
    \end{itemize}

    Pour des simulations sur des données représentatives et l'\textsc{ir}
    français, cf.\ ce papier :

    \url{https://www.ofce.sciences-po.fr/pdf/dtravail/OFCEWP2019-05.pdf}

    Dans le cas de l'imposition séparée, le taux marginal d'imposition du revenu
    des femmes est de \(T'\left(z_{f}\right)\) et dans le cas de l'imposition
    jointe, il est de \(T'\left(\frac{z_{f} + z_{h}}{2}\right)\). Comme
    \(z_{f} < z_{h}\), \( z_{f} < \frac{z_{f}+z_{h}}{2}\). Ainsi,
    \(T'\left(z_{f}\right) < T'\left(\frac{z_{f} + z_{h}}{2}\right)\).

    Reprenons le cas de notre couple inégalitaire. Supposons que la femme
    commence à travailler de telle sorte que son revenu ne soit plus \(0\) mais
    \(10000\).

    L'impôt payé par le couple avec une imposition jointe est alors de
    \(T\left(10000\right) + T\left(40000\right) = 100 + 1600 = 1700\). Pour une
    augmentation de revenu de \(10000\) euros, l'impôt a augmenté de \(100\)
    euros soit 1\% du nouveau revenu.

    Dans le cas de l'imposition jointe, le nouvel impôt est alors de
    \(2 \times T\left(\frac{10000 + 40000}{2}\right) = 2 \times T\left(
    25000\right) = 2 \times 625 = 1250\) euros, soit une augmentation de 450
    euros ou 4,5\% du nouveau revenu.

    Si l'offre de travail des femmes a une élasticité négative par rapport au
    taux marginal d'imposition, elles travailleront moins avec l'imposition
    jointe qu'avec l'imposition séparée, toutes choses égales par ailleurs.

    \dotfill


    Les questions suivantes portent sur l'article \og Are housing benefit an
    effective way to redistribute income? Evidence from a natural experiment
    in France \fg{}, de Gabrielle Fack, paru dans Labour Economics en 2006.

    \textbf{1.} Rappelez ce que sont l'incidence statutaire et l'incidence
    économique d'une taxe ou d'un transfert. Pourquoi diffèrent-elles ?

    \dotfill

    L'incidence statutaire correspond au montant effectivement versé à
    l'institution chargée de collecter les prélèvements obligatoires. On la
    distingue de l'incidence économique qui est le coût effectivement payé par
    les agents, c'est à dire la somme du montant versé par l'agent versé à
    l'administration fiscale et de l'écart de prix causé par la taxe. C'est
    parce que la mise en place d'une taxe modifie le prix d'équilibre que les
    deux ne concordent pas.

    À mon sens, il est utile de distinguer trois incidences :
    \begin{itemize}
        \item l'incidence légale ou statutaire : qui paye la taxe d'après la loi ?
        \item l'incidence comptable : qui verse effectivement une somme d'argent
            à l'administration fiscale ?
        \item l'incidence économique ou fiscale
    \end{itemize}

    La définition \og traditionnelle \fg{} ne distingue pas entre ce que je
    nomme incidence statutaire et incidence comptable.

    Pour prendre l'exemple de l'impôt sur le revenu, la mise en place du
    prélèvement à la source constitue un changement d'incidence comptable, des
    particuliers vers les entreprises, sans changement ni d'incidence légale, ni
    d'incidence économique (a priori).

    L'incidence comptable est importante au moins pour les raisons suivantes :
    \begin{itemize}
        \item les capacités de l'administration fiscale de collecter l'impôt et
            de contrôler la fraude dépendent de l'incidence comptable. Par
            exemple, il serait sans doute très coûteux de collecter les taxes
            sur la consommation auprès des consommateurs plutôt qu'auprès des
            entreprises.
        \item L'agent payant l'impôt au sens comptable n'a pas nécessairement
            les mêmes incitations que l'agent payant l'impôt au sens statutaire
            ou au sens économique. Dans le cas du prélèvement à la source,
            lorsque l'employeur déclare les revenus de son employé à
            l'administration fiscale, il n'a aucun intérêt à frauder ; ce qui
            n'est évidemment pas le cas du contribuable qui déclare ses revenus
            et qui a tout intérêt à les minorer. C'est le cas aussi de la
            taxation des revenus du capital, qui peut se faire au niveau des
            entreprises ou des actionnaires. Dans le cas des \textsc{apl},
            le fait que les bailleurs puissent, dans certains cas, directement
            toucher les \textsc{apl} (et de réduire le loyer d'autant) leur permet
            d'observer directement leur montant et d'ajuster le loyer en
            fonction de cela, ce qui est plus difficile si il n'observe pas
            directement le montant des \textsc{apl}.
        \item Les individus ont une connaissance imparfaite du système
            socio-fiscal. Le fait de payer au sens comptable un impôt le rend
            directement visible. C'est probablement pour cela que le débat public
            accorde une importance bien plus grande à l'impôt sur le revenu
            (jusqu'à présent payé au sens comptable par les contribuables) quà
            la \textsc{csg} (payée au sens comptable par les employeurs et
            institutions financières) alors que la seconde est une taxe sur les
            revenus plus importante en volume que le premier.
    \end{itemize}

    \dotfill

    \textbf{2.} Comment l'autrice mesure-t'elle l'incidence des \textsc{apl} ?
    Quelle variation exploite-t'elle ? Quels sont les groupes de contrôle et
    de traitement ? La méthode employée vous paraît-t'elle pertinente ?

    \dotfill

    L'incidence est calculée à partir de la variation des loyers (= prix) causée
    par les \textsc{Apl}.

    Pour cela, l'autrice exploite une extension de l'égibilité des ménages aux
    aides au début des années 1990.

    Le premier quartile de la distribution des revenus (le quart le plus pauvre
    de la population) compte la majeure partie de ces ménages nouvellement
    éligibles. Il est donc très peu traité avant la réforme et traité après la
    réforme. Ce groupe est le groupe de traitement.

    Le second quartile de la distribution des revenus est déjà en partie traité
    et son niveau de traitement ne varie pas après la réforme. Il s'agit du
    groupe de contrôle.

    En supposant que les loyers payés par les individus de ces deux groupes
    auraient évolué de manière parallèle en l'absence de la réforme, l'autrice
    utilise la méthode de la différence de différences pour mesurer l'effet
    causal de l'extension de l'égibilité des aides au logement sur le niveau des
    loyers.

    \dotfill

    \textbf{3.} Quelle est l'incidence estimée ? Qu'en concluez-vous sur
    l'efficacité des \textsc{apl} pour réduire le montant du loyer des
    bénéficiaires ?

    \dotfill

    Un euro d'\textsc{apl} supplémentaire cause une hausse de loyer de 78
    centimes.

    Les \textsc{apl} permettent donc d'augmenter le revenu disponible des
    personnes éligibles de 22 centimes par euro d'\textsc{apl}.

    Il est cependant probable que d'autres interventions de l'État puisse avoir
    un meilleur rendement (euro de pouvoir d'achat pour les personnes ciblées /
    euro de dépense publique) plus élevé que les aides au logement.

    \dotfill


\end{document}
