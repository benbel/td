\documentclass[../main.tex]{subfiles}

\begin{document}

    \section{}

    \section{Exercices}


    \textbf{1.} Quel est l'effet d'allonger la durée d'indemnisation du chômage d'après Le Barbanchon, 2016 ?
        \begin{enumerate}
            \item Cela améliore la qualité de l'emploi retrouvé et retarde le retour à l'emploi
            \item Cela améliore la qualité de l'emploi retrouvé et ne retarde pas le retour à l'emploi
            \item Cela n'améliore pas la qualité de l'emploi retrouvé et retarde le retour à l'emploi
            \item Cela n'améliore pas la qualité de l'emploi retrouvé et ne retarde pas le retour à l'emploi
	    \end{enumerate}

    \textbf{2.} (Camille Hémet, examen 2017) Pour évaluer l'effet de la durée d'indemnisation par l'assurance chômage sur le
retour à l'emploi, Thomas Le Barbanchon exploite
        \begin{enumerate}
            \item le fait que la durée d'indemnisation potentielle augmente de façon
            discontinue lorsque la durée travaillée dépasse un certain seuil
            \item une variation exogène dans la durée potentielle d'indemnisation induite par
            des différences de salaire
            \item une variation exogène de la durée d'indemnisation potentielle induite par
            une réforme de l'assurance chômage
            \item le fait que des individus ayant des types de contrats différents (CDD, CDI,
            saisonnier, intérimaire) n'ont pas les mêmes durées d'indemnisation
            potentielle
	    \end{enumerate}

    \textbf{1.} Qu'est ce qu'une assurance actuariellement neutre ?

    Dans les exercices précédents, nous avons étudié un individu \(i\), dont le salaire
    est \(s_{i}\) et dont la fonction d'utilité est :
    \[ U_{i} = c_{i} - \frac{h_{i}^{2}}{2} \]

    Supposons désormais que cet individu a une probabilité \(p_{i}\) d'être au
    chômage (auquel cas \(h_{i, \text{ chômage}} = 0\)) et une probabilité 
    \(1 - p_{i}\) de travailler (auquel cas il choisit son offre de travail librement).

    \textbf{1.} Quel est la relation entre  \(\kappa_{i}\) et \(\gamma_{i}\) ?

    \textbf{2.} L'utilité espérée de l'individu dépend-elle du niveau d'assurance \(\left(\kappa_{i}, \gamma_{i} \right)\) ?

    \textbf{3.} L'individu choisirait-il de souscrire à une assurance chômage non actuariellement neutre ?
    Pourquoi ?

   \textbf{B.} Assurance \textit{3 points}

    \textbf{1.} Qu'est ce qu'une assurance actuariellement neutre ?

    \dotfill

    C'est une assurance dont la prime de risque est égale à l'espérance des paiements,
    i.e. en moyenne le coût et le bénéfice pour l'assuré comme pour l'assurance sont nuls.

    \dotfill

    Dans les exercices précédents, nous avons étudié un individu \(i\), dont le salaire
    est \(s_{i}\) et dont la fonction d'utilité est :
    \[ U_{i} = c_{i} - \frac{h_{i}^{2}}{2} \]

    Supposons désormais que cet individu a une probabilité \(p_{i}\) d'être au
    chômage (auquel cas \(h_{i, \text{ chômage}} = 0\)) et une probabilité 
    \(1 - p_{i}\) de travailler (auquel cas il choisit son offre de travail librement).

    Supposons que cet agent puisse acheter une assurance acturiellement neutre, dont
    le coût est \(\kappa_{i}\) lorsqu'il est employé et qui lui verse \(\gamma_{i}\) lorsqu'il est au chômage.

    \textbf{1.} Quel est la relation entre  \(\kappa_{i}\) et \(\gamma_{i}\) ?

    \dotfill

    La prime de risque est de \(\kappa_{i}\), payée avec une probabilité \(p_{i}\). La prime de risque espérée est
    donc de \(\kappa_{i} \times p_{i}\).

    Le paiement en cas de survenue du risque est de \(\gamma_{i}\), payé avec une probabilité \(1 - p_{i}\). Le paiement espéré
    est donc de \(\gamma_{i} \times \left(1-p_{i}\right)\).

    Puisque l'assurance est actueriellement neutre, on a : \[\kappa_{i} \times p_{i} = \gamma_{i} \times \left(1-p_{i}\right) \]

    C'est à dire : \[\kappa_{i} = \gamma_{i} \times \frac{1-p_{i}}{p_{i}} \]

    \dotfill

    \textbf{2.} L'utilité espérée de l'individu dépend-elle du niveau d'assurance \(\left(\kappa_{i}, \gamma_{i} \right)\) ?
    
    \dotfill
    
    Si l'individu ne travaille pas, on a \(c_{i} = \gamma_{i}\) et \(h_{i} = 0\). Dès lors, \(U_{i}^{c} = \gamma_{i} \) où \(U_{i}^{c}\) est l'utilité de l'individu lorsqu'il est au chômage.
    
    Si l'individu travaille, on a \(c_{i} = s \times h_{i} - \kappa_{i}\). L'individu maximise son utilité
    en choisissant son nombre d'heures travaillées, et alors \(h_{i}^{*} = s\) et \(U_{i}^{t} = \frac{s^{2}}{2} - \kappa_{i} \) où \(U_{i}^{t}\) est l'utilité de l'individu lorsqu'il travaille. 
    
    Son utilité espérée est donc :
    \begin{align*}
    \left(1-p_{i}\right) \times U_{i}^{c} + p_{i} \times U_{i}^{t} &= \left(1-p_{i}\right) \gamma_{i} + p_{i} \times \left( \frac{s^{2}}{2} - \kappa_{i} \right) \\
    &= \left(1-p_{i}\right) \gamma_{i} - p_{i} \kappa_{i} + \frac{p_{i}s^{2}}{2} \\
    &= 0 + \frac{p_{i}s^{2}}{2} \\
    &=  \frac{p_{i}s^{2}}{2}
    \end{align*}
    
    L'utilité de l'individu ne dépend pas de son niveau d'assurance.
    
    \dotfill


    \textbf{3.} L'individu choisirait-il de souscrire à une assurance chômage non actuariellement neutre ?
    Pourquoi ?
    
    \dotfill
    Non, car cela diminuerait son utilité espérée : \(\left(1-p_{i}\right) \gamma_{i} - p_{i} \kappa_{i} < 0\). L'individu est neutre face au risque, et n'est donc pas prêt à payer pour lisser sa consommation.
    
    \dotfill
\end{document}
