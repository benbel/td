\documentclass[11pt, a4paper]{article}

\usepackage[french]{babel}

\usepackage[utf8]{inputenc}
\usepackage[T1]{fontenc}

\usepackage{xcolor}

\usepackage{amsmath}
\usepackage{amssymb}
\usepackage{amsfonts}
\usepackage{eurosym}

\usepackage{graphicx}

\usepackage{tikz}

\usepackage{hyperref}
\hypersetup{
    colorlinks=true,
    linkcolor=red,
    }

\usepackage[]{geometry}
\setlength{\parindent}{0cm}
\setlength{\parskip}{.5\baselineskip}
\linespread{1.3}

\begin{document}

    \section*{Exercices sur l'économie du bien-être}

	\subsection*{Amélioration au sens de Pareto et boîte de Edgeworth}

    \textbf{1.} Si pour deux agents, A et B, et deux biens, j et k, la fonction
    d'utilité est $u_a=u_b=j*k$ et l'allocation initiale est $ e_a=(5,3)$ et $e_b=(5,7)$,
    quelle allocation est améliorante au sens de Pareto ?
		\begin{enumerate}
		\item $e_a = (5,8)$ et $e_b = (5,2)$
		\item $e_a = (6,6)$ et $e_b = (4,4)$
		\item $e_a = (6,5)$ et $e_b = (4,5)$
		\item $e_a = (4,4)$ et $e_b = (6,6)$
		\end{enumerate}

    \begin{center}
    \begin{tikzpicture}
        \draw (0,10) node[left] {100} -- (0,0) node[midway,left] {Bien Y} node[left]{A} -- (10,0) node[midway,below] {Bien X} node[below] {100};
        \draw (0,10) node[left] {100} -- (10,10) node[midway,above] {Bien X} node[right]{B} -- (10,0) node[midway,right] {Bien Y} node[below] {100};

        \draw[very thick] (1, 9) to[out=0,in=90] (9, 1);
        \draw[very thick] (1, 6.63) to[out=0,in=90] (6.63, 1);
        \draw[thick,dashed] (1, 7.5) to[out=0,in=90] (7.5, 1);
        \draw[thick,dashed] (1, 8.25) to[out=0,in=90] (8.25, 1);
        \draw[thick,dashed,red] (3.33, 9) to[out=-90,in=180] (9, 3.33);

        \draw[very thick,red] (1, 9) to[out=-90,in=180] (9, 1);
        \draw[thick,dashed,red] (1.8, 9) to[out=-90,in=180] (9, 1.8);
        \draw[thick,dashed,red] (2.5, 9) to[out=-90,in=180] (9, 2.5);
        \draw[very thick,red] (3.33, 9) to[out=-90,in=180] (9, 3.33);

        \draw[fill=blue] (1,9) circle (1mm) node[left=2pt] {a};
        \draw[fill=blue] (9,1) circle (1mm) node[right=2pt] {b};
        \draw[fill=blue] (5,5) circle (1mm) node[right=2pt] {c};
        \draw[fill=blue] (8.85,2.5) circle (1mm) node[right=2pt] {d};
        \draw[fill=blue] (1.8,8.95) circle (1mm) node[above=2pt] {e};

    \end{tikzpicture}
    \end{center}

    \textbf{2.} Dans la boîte de Edgeworth ci-dessus, quelles sont les allocations
        Pareto-efficaces ?
        \begin{enumerate}
            \item Il n'y a pas d'allocation Pareto-efficace représentée
            \item \textsc{a} et \textsc{b}
            \item \textsc{c}
            \item \textsc{d}
	    \end{enumerate}


    \textbf{3.} Dans la boîte de Edgeworth ci-dessus, quelles sont les allocations
        Pareto-améliorantes par rapport à \textsc{e} ?
        \begin{enumerate}
            \item \textsc{a}, \textsc{d} et \textsc{b}
            \item \textsc{c}
            \item \textsc{d}
            \item \textsc{c} et \textsc{d}
	    \end{enumerate}

    \textbf{4.} Dans une boîte de Edgeworth, que représente la courbe des contrats ?
	    \begin{enumerate}
	    \item Les allocations Pareto-améliorantes
	    \item Les allocations Pareto-efficaces
	    \item Les courbes d'indifférence des deux agents
	    \item Les allocations qui maximisent le bien-être social
	    \end{enumerate}


	\textbf{5.} Le critère de Pareto permet-il de classer entre elles toutes
        les situations possibles ?

    \subsection*{\textsc{tms}} 

    On réprésente les préférences d'un individu par la fonction d'utilité
        \(U\left(x,y\right) = x^{2} + y^{2}\). Quel est le taux marginal de substitution
        du bien $x$ en termes du bien $y$ pour cet agent ? On raisonne ici en valeur absolue,
        c'est-à-dire avec un \textsc{tms} positif.
		\begin{enumerate}
		\item $\frac{x}{y}$
		\item $2 \frac{x}{y}$
		\item $y$
		\item $2x$
		\end{enumerate}

    \subsection*{Théorèmes du bien-être}
     Laquelle des propositions suivantes énonce le premier théorème du bien-être ?
	    \begin{enumerate}
	    \item Toute allocation Pareto optimale est efficace.
	    \item Toute allocation d'équilibre concurrentiel est un optimum de Pareto.
	    \item Toute allocation d'équilibre concurrentiel peut être rendue équitable
              après avoir réalisé les transferts adéquats.
	    \item Toute allocation Pareto optimale peut être obtenue comme équilibre
        concurrentiel après réallocation adéquate des dotations initiales.
	    \end{enumerate}

    \subsection*{Edgeworth}

    On étudie une économie avec deux agents, \(A\) et \(B\), et deux biens, \(\gamma\)
    et \(\delta\).

    La quantité totale du bien \(\gamma\) est de 5 unités et celle du bien \(\delta\)
    est de 10 unités.

    On note \(\gamma_{A}\) la quantité du bien \(\gamma\) détenue par l'individu
    \(A\), \(\delta_{A}\) la quantité du bien \(\delta\) détenue par l'individu
    \(A\), \(\gamma_{B}\) la quantité du bien \(\gamma\) détenue par l'individu
    \(B\), et \(\delta_{B}\) la quantité du bien \(\delta\) détenue par l'individu
    \(B\).

    Les deux agents ont pour fonction d'utilité respectivement \(U_{A}\left(\gamma_{A},
    \delta_{A}\right) = \gamma_{A} + \delta_{A}^{1/2}\) et
    \(U_{B}\left(\gamma_{B}, \delta_{B}\right) = \gamma_{B}^{1/2} + \delta_{B}\)


    L'allocation initale des biens est de 3 unités du bien \(\gamma\) et 4
    unités du bien \(\delta\) pour \(A\), que l'on note \(A = \left(3,4\right)\),
    et de \(B = \left(2,6\right)\).

    \textbf{1.} Représentez cette économie par une boîte de Edgeworth

    \textbf{2.} Représentez les courbes d'indifférence sur lesquels se placent
    les deux individus

    \textbf{3.} L'allocation initiale est-elle Pareto-efficace ?
    Si non, quelles sont les allocations Pareto-améliorantes ?

    \textbf{4.} Déterminez une allocation Pareto-efficace. 
    Calculez le taux marginal de substitution des deux agents en ce point.
    Quelle propriété satisfont-ils ?

    \textbf{5.} L'allocation initiale respecte-t'elle le principe rawlsien du maximin ?

    \subsection*{SWF}
    
    On étudie une économie de 10 agents, numérotés de 1 à 10, dont les revenus
    du travail sont :

    \begin{table}[h!]
        \centering
        \begin{tabular}{c|cccccccccc}
            Individu & 1 & 2 & 3 & 4  & 5  & 6  & 7  & 8  & 9  & 10 \\ \hline
            Revenu   & 0 & 0 & 5 & 10 & 10 & 20 & 30 & 40 & 50 & 85
        \end{tabular}
    \end{table}

    \textbf{1.} Quel est le revenu moyen ? Quel est le revenu médian ?

    \textbf{2.} Tracez la courbe de Lorenz pour le revenu dans cette économie.

    Supposez que tous les agents ont une fonction d'utilité
    \(U\left(R\right) = R^{1/2}\).

    \textbf{3.} Quels transferts faudrait-il mettre en place pour maximiser une
    fonction de bien-être social utilitariste ?

\end{document}
