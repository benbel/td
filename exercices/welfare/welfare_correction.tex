\documentclass[11pt, a4paper]{article}

\usepackage[french]{babel}

\usepackage[utf8]{inputenc}
\usepackage[T1]{fontenc}

\usepackage{xcolor}

\usepackage{amsmath}
\usepackage{amssymb}
\usepackage{amsfonts}
\usepackage{eurosym}

\usepackage{graphicx}

\usepackage{tikz}

\usepackage{hyperref}
\hypersetup{
    colorlinks=true,
    linkcolor=red,
    }

\usepackage[]{geometry}
\setlength{\parindent}{0cm}
\setlength{\parskip}{.5\baselineskip}
\linespread{1.3}

\begin{document}

    \section*{Correction des  sur l'économie du bien-être}

	\subsection*{Amélioration au sens de Pareto et boîte de Edgeworth}

    \textbf{1.} Pour qu'une allocation soit améliorante au sens de Pareto, il faut (par
    définition de ce qu'est une amélioration au sens de Pareto) que l'utilité
    des deux agents soit au moins aussi élevée que dans l'allocation initiale
    (personne n'est lésé) et que l'utilité d'au moins un des deux agents soit
    plus élevée.

    Dans l'allocation initiale, \(u_a = 5 * 3 = 15\) et \(u_b = 5*7 = 35\)

    Autrement, on cherche une allocation telle que :
    \begin{itemize}
        \item \(u_a = 15\) et \(u_b > 35\),
        \item ou \(u_a > 15\) et \(u_b = 35\)
        \item ou \(u_a > 15\) et \(u_b > 35\)
    \end{itemize}

    L'allocation \(4\) est Pareto améliorante, puisque \(u_a = 16 > 15\) et
    \(u_b = 36 > 35\).

    \textbf{2.} Seule l'allocation \(c\) est efficace au sens de Pareto.

    Graphiquement, les allocations qui améliorent l'utilité de l'agent \textit{A}
    par rapport à une allocation donnée se situent \og{}en haut à droite\fg{}
    (i.e.\ plus loin de l'origine pour \textit{A})  de la courbe d'indifférence de l'individu
    \textit{A} sur laquelle se trouve l'allocation en question.

    De la même manière, les allocations qui améliorent l'utilité de l'agent
    \textit{B} par rapport à une allocation donnée se situent
    \og{}en bas à gauche\fg{} (i.e.\ plus loin de l'origine pour \textit{B})
    de la courbe d'indifférence de l'individu \textit{B} sur laquelle se trouve
    l'allocation en question.

    Les allocations améliorantes au sens de Pareto se trouvent donc dans
    l'intersection de ses deux surfaces, c'est à dire dans la lentille formée
    par les courbes d'indifférence de \textit{A} et \textit{B} qui passent par
    l'allocation.

    Une allocation est efficace au sens de Pareto si il n'existe pas
    d'amélioration au sens de Pareto ; ainsi, la lentille est vide, i.e.\ les
    deux courbes d'indifférence sont tangentes l'une à l'autre.

    \textbf{3.} L'allocation \(c\) se situe sur une courbe d'indifférence
    correspondant à une utilité supérieure pour \textit{A} comme pour \textit{B}
    : elle est donc Pareto-améliorante par rapport à \(e\). L'allocation \(d\)
    ne dimininue ni n'augmente l'utilité de \textit{B} (même courbe
    d'indifférence) ; elle augmente celle de \textit{A} : c'est donc également
    une Pareto-amélioration.

    \textbf{4.} L'ensemble des allocations efficaces.

    \textbf{5.} Non, puisqu'il existe une infinité d'allocations efficaces.

    \subsection*{TMS}

        \[TMS_{x,y} = \frac{\frac{\partial U}{\partial x}}{\frac{\partial U
        }{\partial y}} = \frac{2x}{2y} = \frac{x}{y} \]
    \textbf{1.} Si pour deux agents, A et B, et deux biens, j et k, la fonction
    d'utilité est $u_a=u_b=j*k$ et l'allocation initiale est $ e_a=(5,3)$ et $e_b=(5,7)$,
    quelle allocation est améliorante au sens de Pareto ?
		\begin{enumerate}
		\item $e_a = (5,8)$ et $e_b = (5,2)$
		\item $e_a = (6,6)$ et $e_b = (4,4)$
		\item $e_a = (6,5)$ et $e_b = (4,5)$
		\item $e_a = (4,4)$ et $e_b = (6,6)$
		\end{enumerate}


    \subsection*{Edgeworth}

    \textbf{1 \& 2} Cf.\ graphes.

    \begin{figure}[p]
        \centering
        \includegraphics[width=8cm]{allocation_initiale}
        \includegraphics[width=8cm]{indifference_a}
        \includegraphics[width=8cm]{indifference_b}
        \caption{Représentation graphique de l'économie}
    \end{figure}

    \textbf{3.} Non : toutes les allocations dans la lentille en haut à gauche de
    l'allocation initiale sont des améliorations au sens de Pareto :
    \begin{itemize}
        \item le long de la courbe d'indifférence de \(A\) (en orange),
            l'utilité de \(A\) est inchangée et celle de \(B\) est plus élevée
        \item le long de la courbe d'indifférence de \(B\) (en bleu),
            l'utilité de \(B\) est inchangée et celle de \(A\) est plus élevée
        \item entre les deux courbes, l'utilité des deux agents augmente
    \end{itemize}

    \textbf{4.} Trivialement, les allocations où tous les biens sont possédés par \(A\) ou
    par \(B\) sont Pareto-efficaces.

    On peut également partir de la définition de l'équilibre (la demande totale
    pour chacun des deux biens est égale à l'offre) ou d'une de ses propriétés
    (les taux marginaux de substitution sont égaux à l'équilibre) pour trouver
    un équilibre : or, comme toutes les équilibres concurrentiels sont efficaces
    au sens de Pareto, cela correspond à une allocation Pareto-efficace.

    Partons de l'allocation \(A = \left(4, 1\right)\) et \(B = \left(1, 9\right)\)
    (pour se simplifier les calculs
    et notamment éviter les cas particuliers où la consommation optimale d'un bien est
    négative). Le raisonnement est le même quelque soit l'allocation de départ.

    Notons \(p_{\delta}\) le prix d'équilibre du bien \(\delta\) exprimé en unités
    du bien \(\gamma\). Notons \(p_{\gamma}\) le prix d'équilibre du bien \(\gamma\)
    exprimé en unités du bien \(\gamma\). On a donc \(p_{\gamma} = 1\). On appelle
    alors le bien \(\gamma\) le numéraire.

    Plus généralement, si \(\left(p'_{\delta}, p'_{\gamma}\right)\) sont les prix
    d'équilibre, alors pour tout \(\Phi > 0\), \(\left(p'_{\delta} / \Phi, p'_{\gamma} / \Phi \right)\)
    sont également des prix d'équilibre. Par souci de clarté, on fixe un des prix
    à 1 (le prix du bien numéraire). Ici, on prend \(\Phi = p'_{\gamma}\) et le vecteur
    des prix d'équilibre est donc \(\left(p'_{\delta} / p'_{\gamma}, p'_{\gamma} / p'_{\gamma} \right) = \left(p_{\delta}, 1\right)\).

    Compte tenu de ces prix d'équilibre et de l'allocation initiale, la valeur
    des biens détenus par \(A\) est :
    \[ R_{A} = \gamma^{\text{initial}}_{A} \times p_{\gamma} + \delta^{\text{initial}}_{A} \times p_{\delta} = 4 \times 1 + 1 \times p_{\delta} = 4 + p_{\delta} \]

    La valeur de son panier de consommation est :
    \[ \gamma_{A} \times p_{\gamma} + \delta_{A} \times p_{\delta} = \gamma_{A}  + \delta_{A} \times p_{\delta} \]

    À l'optimum, la contrainte de budget sera saturée :
    \[ \gamma_{A}  + \delta_{A} \times p_{\delta}  = 4 + p_{\delta} \Rightarrow \gamma_{A} = 4 + p_{\delta} - \delta_{A} p_{\delta} \]

    L'individu \(A\) maximise son utilité :
    \[ \max U_{A} = \gamma_{A} + \delta_{A}^{1/2} =  4 + p_{\delta} - \delta_{A} p_{\delta} + \delta_{A}^{1/2}\]

    Au maximum, la dérivée de \(U_{A}\) par rapport à la variable de choix (\(\delta_{A}\)) est nulle :
    \begin{multline*}
         \frac{dU_{A}}{d\delta_{A}} = - p_{\delta} + \frac{1}{2\sqrt{\delta_{A}}} = 0 \\
        \Rightarrow \delta_{A} = \frac{1}{4p_{\delta}^{2}} \\
        \Rightarrow \gamma_{A} = 4 + p_{\delta} - \frac{1}{4p_{\delta}}
    \end{multline*}

    On procède de la même manière pour l'individu \(B\) (i.e.\ on exprime son
    revenu, sa contrainte de budget et on maximise son utilité sous containte), et
    l'on obtient :
    \begin{align*}
        \gamma_{B} &= \frac{1}{4} p_{\delta}^{2}\\
        \delta_{B} &= \frac{1}{p_{\delta}} - \frac{1}{4} p_{\delta} + 9\\
    \end{align*}

    Nous avons donc les offres totales en \(\gamma\) et en \(\delta\) (qui sont
    fixes et données dans l'énoncé), ainsi que les demandes totales en \(\gamma\)
    et en \(\delta\), qui sont respectivement \(\gamma_{A} + \gamma_{B}\) et
    \(\delta_{A} + \delta_{B}\).

    À l'équilibre (par définition de ce qu'est l'équilibre), l'offre est égale
    à la demande pour chacun des biens :
    \begin{align*}
        5 &= \gamma_{A} + \gamma_{B} &=  4 + p_{\delta} - \frac{1}{4p_{\delta}} + \frac{1}{4p_{\delta}^{2}} \\
        10 &= \delta_{A} + \delta_{B} &=  \frac{1}{4p_{\delta}^{2}} +  \frac{1}{p_{\delta}} - \frac{1}{4} p_{\delta} + 9
    \end{align*}

    À partir de ces équations, il est possible de calculer le prix \(p_{\delta}\) d'équilibre.

    Partons de la seconde équation :
    \begin{align*}
        & 10 = \frac{1}{4p_{\delta}^{2}} +  \frac{1}{p_{\delta}} - \frac{1}{4} p_{\delta} + 9 \\
        \Rightarrow & 1 =  \frac{1}{4p_{\delta}^{2}} +  \frac{1}{p_{\delta}} - \frac{1}{4} p_{\delta} \\
        \Rightarrow & \frac{1}{4p_{\delta}^{2}} +  \frac{1}{p_{\delta}} - \frac{1}{4} p_{\delta} - 1 = 0 \\
        \Rightarrow & \frac{1}{4} +  p_{\delta} - \frac{1}{4} p_{\delta}^{3} - p_{\delta}^{2} = 0 \\
        \Rightarrow & 1 +  4 p_{\delta} -  p_{\delta}^{3} - 4 \times p_{\delta}^{2} = 0 \\
    \end{align*}

    À partir de la dernière ligne, on peut résoudre à la main l'équation ou \href{https://www.wolframalpha.com/input/?i=solve+1+%2B+4x+-+x%5E3+-+4x%5E2+%3D+0}{via
    son ordinateur} et on trouve \(p_{\delta} = 1\).

    On peut en déduire toutes les quantités d'équilibre :
    \begin{align*}
        \gamma_{A} &= 4 + p_{\delta} - \frac{1}{4p_{\delta}} &= 4 + 1 - \frac{1}{4} &= 5 - \frac{1}{4} \\
        \delta_{A} &= \frac{1}{4p_{\delta}^{2}}  &= \frac{1}{4}\\
        \gamma_{B} &= \frac{1}{4} p_{\delta}^{2} &= \frac{1}{4}\\
        \delta_{B} &= \frac{1}{p_{\delta}} - \frac{1}{4} p_{\delta} + 9 &= 1 - \frac{1}{4} + 9 &= 10 - \frac{1}{4}
    \end{align*}

    D'après le premier théorème du bien-être l'allocation \(A = \left( 5 - \frac{1}{4}, \frac{1}{4} \right) \) et
    \( B = \left( \frac{1}{4}, 10 - \frac{1}{4} \right) \) est Pareto-efficace.

    On peut en déduire les taux marginaux de substitution en ce point :
    \begin{align*}
        TMS_{\gamma,\delta}^{A} &= \frac{\partial U_{A}}{\partial \gamma_{A}} / \frac{\partial U_{A}}{\partial \delta_{A}} = \frac{1}{\frac{1}{2\sqrt{\frac{1}{4}}}} = 2 \sqrt{\frac{1}{4}} = 2 \times \frac{1}{2} &= 1 \\
        TMS_{\gamma,\delta}^{B} &= \frac{\partial U_{B}}{\partial \gamma_{B}} / \frac{\partial U_{B}}{\partial \delta_{B}} = \frac{\frac{1}{2\sqrt{\frac{1}{4}}}}{1} = \frac{1}{2\sqrt{\frac{1}{4}}} = \frac{1}{2 \times \frac{1}{2}} &= 1 \\
        \frac{p_{\gamma}}{p_{\delta}} &= \frac{1}{1} &= 1
    \end{align*}

    Au point d'équilibre qui est aussi une allocation Pareto-efficace, les taux
    marginaux de substitution des deux agents sont égaux et égaux au ratio des
    prix. Graphiquement, les deux courbes d'indifférence et la droite de budget sont tangentes.

    \textbf{5.} L'utilité de l'individu qui a la plus basse utilité (\(A\)) est de 5.

    Pour l'allocation (par exemple) \(A = \left(5,1\right)\) et \(B =
    \left(0,9\right)\), l'utilité de l'individu qui a la plus basse utilité
    (toujours \(A\)) est maintenant de 6.

    L'allocation initiale ne maximise donc pas l'utilité de l'individu le moins
    bien loti.

    \subsection*{Théories de la justice sociale}

    \textbf{1.}
        \begin{itemize}
        \item Impossibilité pratique de mesurer l'utilité
        \item Poids identique accordé à tous les individus indépendemment de
            leurs caractéristiques (par exemple, pas de considération
            particulière pour les plus pauvres en tant que tels)
        \item Pas de jugement sur la manière dont l'utilité est procurée, i.e.\
            deux activités qui procurent la même utilité sont jugées aussi
            bonnes (e.g.\ persécuter une minorité peut donc être optimal)
        \item Pas de considération pour les droits fondamentaux en dehors de
            l'utilité qu'ils pourraient apporter
    \end{itemize}

    \textbf{2.} Oui, tant que cette distribution améliore le sort du moins bien loti.
    Une allocation qui améliore le sort du moins bien loti mais augmente les
    inégalités est donc positive au sens du maximin.

    \subsection*{SWF}

    \textbf{1.} Le revenu moyen est de 25 et le revenu médian (revenu tel que la moitié des
    individus ont plus que ce revenu et l'autre moitié a moins) est de 15.

    \textbf{2.} Cf.\ graphe.

    \begin{figure}[p]
        \centering
        \includegraphics[width=8cm]{lorenz}
        \caption{Courbe de Lorenz de cette économie}
    \end{figure}

    \textbf{3.} Soient deux agents \(A\) et \(B\) tels que \(R_{A} > R_{B}\).

    Alors, l'utilité marginale de \(A\) est plus faible que celle de \(B\) :
    \[ \frac{dU_A}{R_{A}} = \frac{1}{2\sqrt{R_{A}}} < \frac{1}{2\sqrt{R_{B}}} = \frac{dU_B}{R_{B}} \]

    Ainsi, lorsque l'on transfert du revenu de \(A\) à \(B\) on modifie la somme \(U_A + U_B\) (et
    donc la fonction de bien-être social) de \( - \frac{dU_A}{R_{A}} + \frac{dU_B}{R_{B}} > 0\).

    Autrement dit, tout transfert des plus riches vers les plus pauvres augmente la somme
    des utilités.

    La fonciton de bien-être social est donc maximisée quand il n'est plus possible de l'augmenter,
    i.e.\ quand il n'y a plus de transfert des riches vers les pauvres possibles, i.e.\ lorsque
    tous les individus ont le même revenu.

    Il faut donc appliquer un transfert de \(25 - R_{i}\) à tous les individus, de telle sorte
    que le revenu de chacun soit, après transfert \(R_{i} + \left(25 - R_{i}\right) = 25\).

\end{document}

