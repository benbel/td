\documentclass[../main.tex]{subfiles}

\begin{document}

    \subsection{}

    \subsection{Exercices}

    On étudie une population de \(N\) individus identiques, dont l'utilité
    dépend de leur temps de travail \(T\) et de leur consommation \(C\) :
    \[ U\left(C,T\right) = C - \frac{1}{2} T^{2} \]

    On suppose que le prix du bien de consommation est de 1, et
    qu'une unité de travail est rémunérée par un salaire \(s\).

    On suppose que chaque individu maximise son utilité.

    \textbf{1.} Quel est le niveau de consommation choisi par chaque individu ?
    Quelle est leur offre de travail ? Quelle est alors l'offre de travail totale ?


    Soient \(M\) entreprises identiques qui produisent un bien \(y\) dont le
    prix est \(p_{y} = 1\) à l'aide de travail, dont le coût unitaire est \(s\).
    La fonction de production de l'entreprise est :
    \[ y = f\left(T\right) = T - \frac{1}{2}T^{2} \]

    On suppose que l'entreprise maximise son profit.

    On suppose que le marché est en concurrence pure et parfaite.

    \textbf{2.} Quelle est la quantité de travail demandée par les entreprises
    pour maximiser leur profit ?

    \textbf{3.} Représentez graphiquement les courbes d'offre et de demande de
    travail.

    \textbf{4.} Quel est le salaire d'équilibre \(s^{*}\) et la quantité de
    travail d'équilibre \(q^{*}\) sur le marché du travail ?

    \textbf{5.} Comment le salaire d'équilibre évolue-t-il avec le nombre \(N\)
    d'individus et avec le nombre \(M\) d'entreprises ? Expliquez l'intuition du
    résultat.

    Le Gouvernement introduit un salaire minimum \(s_{min}\).

    \textbf{6.} À l'aide de votre graphique, déterminez l'effet de cette
    mesure sur le marché du travail.
    Distinguez le cas où \(s_{min}\) est inférieur au salaire
    d'équilibre sans salaire minimum et le cas où \(s_{min}\) est supérieur
    au salaire d'équilibre sans salaire minimum.

    On ne suppose plus que le marché est en concurrence pure et parfaite ; on
    considère le cas d'un monopsone sur le marché du travail, avec une seule
    entreprise dont la fonction de production est \(y = g\left(T\right) =
    M\left(T - \frac{1}{2}T^{2}\right)\).

    \textbf{7.} Si l'entreprise veut acheter \(T^{m}\) unités de travail, quel
    salaire \(s^{m}\) doit-elle payer pour que les individus offrent cette
    quantité de travail ?

    \textbf{8.} Déduisez-en le profit de l'entreprise en fonction de
    \(T^{m}\). Quelle est la quantité de travail \(T^{m*}\) demandée par
    l'entreprise qui maximise son profit ? Quelle est la valeur de \(s^{m*}\) ?

    \textbf{9.} Comparez \(s^{m*}\) à \(s^{*}\). Lequel est le plus élevé et
    pourquoi ?

    \textbf{10.} Quel serait l'effet d'introduire un salaire minimum compris
    entre \(s^{*}\) et \(s^{m*}\) sur la quantité de travail échangée à
    l'équilibre ?

    Les questions suivantes portent sur l'article \og Minimum Wages and
    Employment: A Case Study of the Fast-Food Industry in New Jersey and
    Pennsylvania \fg{}, de David Card et Alan Krueger, paru dans l'American
    Economic Review en 1994.

    \textbf{11.} De quelle politique publique les auteurs souhaitent-ils mesurer
    les effets ? Quels en sont les effets attendus ?

    \textbf{12.} Expliquez en quoi consiste la méthode des \og différences de
    différences \fg{} utilisée par les auteurs. Sur quelles hypothèses
    repose-t-elle ? Sont-elles vérifiées ?

    \textbf{13.} Que concluent les auteurs sur l'effet de cette politique ?

    \subsection{Correction}

    \textbf{1. Concurrence pure et parfaite}

    Comme l'on est en concurrence pure et parfaite, les individus et les entreprises
    prennent les prix, en l'occurence le prix d'une unité de travail \(s\) comme
    donné.

    \textit{Comportement des individus.} Ils maximisent leur utilité. Ils choisissent
    donc leur niveau de travail \(T\) et de consommation \(C\) qui maximise leur utilité.

    La contrainte budgétaire de l'indididu est \( 1 \times C = s \times T \).

    On peut substituer la contrainte de budget \(C = sT\) dans l'utilité, et
    trouver le maximum de l'utilité en la dérivant par rapport à \(T\) (ou \(C\)).

    Exprimons, par exemple, l'utilité en fonction de \(T\) :
    \begin{align*}
        U\left(C,T\right) & = U\left(sT,T\right) \\
                          & = sT -  - \frac{1}{2} T^{2}
    \end{align*}

    On peut alors calculer la dérivée de \( U \) : \[ \frac{dU}{dT} = s - T \]

    À l'optimum, cette dérivée est nulle, i.e. le \(T^{*}\) choisi par l'individu
    est tel que \[\frac{dU}{dT}\Bigr|_{T^{*}} = 0 \Rightarrow s - T^{*} = 0
    \Rightarrow T^{*} = s\]

    Chaque individu offre donc \(T^{*} = s\) unités de travail, et consomme \(C^{*} = sT = s^2\).

    Comme il y a \(N\) individus identiques dans notre économie, l'offre totale de travail que
    l'on notera \(OT\) est \(OT = Ns\).

    \textit{Comportement des entreprises.} Elles maximisent leur profit. Elles choisissent
    donc leur niveau de travail \(T\) qui maximise leur profit.

    Le profit \( \Pi \) de l'entreprise est égal à :
    \[ \Pi = \underbrace{1 \times \left(  T - \frac{1}{2}T^{2} \right)}_{\text{La
    quantité de bien } y \text{ produite fois son prix de vente}} -
    \underbrace{sT}_{\text{Le coût du facteur de production}} \]

    À l'optimum, la dérivée s'annule :
    \[ \frac{d\Pi}{dT} = 1 - T^{*} - s = 0 \]

    Chaque entreprise demande donc \(T^{*} = 1-s\) unités de travail, et produit
    \(f\left(T^{*}\right) = T^{*} - \frac{1}{2}{T^{*}}^{2} = 1 - s - \frac{1}{2}{\left(1-s\right)}^{2} =
    \frac{1}{2} - \frac{1}{2}{s}^{2} \) unités du bien \(y\).

    Comme il y a \(M\) entreprises identiques dans notre économie, la demande totale de travail que
    l'on notera \(DT\) est \(DT = M\left(1-s\right)\).

    \textit{Équilibre sur le marché du travail.} L'équilibre est la situation où
    la demande est égale à l'offre, \( OT = DT \).

    En notant \(s_{eq}\) le salaire d'équilibre, on a :
    \begin{multline*}
        OT = DT \\
        \Rightarrow Ns_{eq} = M\left(1-s_{eq}\right) \\
        \Rightarrow Ns_{eq} = M - Ms_{eq} \\
        \Rightarrow Ns_{eq} + Ms_{eq} = M  \\
        \Rightarrow \left(N+M\right) s_{eq} = M  \\
        \Rightarrow s_{eq}= \frac{M}{N+N}
    \end{multline*}

    Pour ce niveau de salaire, la quantité de travail échangée sur le marché du
    travail est :
    \[q_{eq} = OT = DT = Ns_{eq} = M\left(1-s_{eq}\right) = \frac{MN}{N+M} \]

    \begin{figure}[p]
        \centering
        \includegraphics[width=10cm]{offre_demande}
        \caption{Représentation graphique du marché du travail pour \(M = 150\) et \(N = 50\)}
    \end{figure}

    Graphiquement, l'équilibre est le point où les courbes d'offre et de
    demande se croisent.

    Lorsque \(M\) augmente, le salaire augmente également (la demande de travail
    augmentant toute choses égales par ailleurs) et tend vers 1 (salaire maximal
    au delà duquel le profit d'une entreprise serait négatif).

    Lorsque \(N\) augmente, pour une demande de travail donnée, chaque individu
    travaille moins. La désutilité au travail étant croissante en \(T\), les
    individus demandent une compensation plus faible par heure de travail
    lorsqu'ils travaillent moins.

    \textbf{2. Monopsone}

    En situation de concurrence pure et parfaite, l'effet de l'offre ou de la
    demande d'un individu ou d'une entreprise est négligeable : c'est pour cela
    que l'individu comme l'entreprise considère le prix comme fixé et ne prend pas
    en compte l'effet de sa demande ou son offre sur le prix.

    En monopsone, la demande de travail par l'entreprise a un effet sur les prix ;
    lorsqu'elle maximise son profit.

    Si elle embauche une quantité \(q\) de travail, elle devra payer un salaire
    de \(q/N\). Si elle payait un salaire inférieur, l'offre de travail totale serait
    inférieure à \(q\).

    Le profit du monopsone est alors :
    \begin{align*}
        \Pi &= M p_{y} f(q) - sq \\
            &= M \left(q - \frac{1}{2} q^{2}\right) - \frac{q}{N} q
    \end{align*}

    La différence fondamentale entre le monopsone et la concurrence pure et
    parfaite est que l'entreprise prend en compte le fait que le salaire n'est pas
    fixe mais une fonction de sa demande de travail : \( s\left(q\right) = \frac{q}{N} \).

    On obtient la demande de travail effectivement demandée par l'entreprise en
    maximisant son profit :
    \[ \frac{d\Pi}{dq} = M - Mq^{*} - \frac{2}{N} q^{*} = 0 \]

    On trouve :
    \[q^{*} = OT = \frac{MN}{MN +2} = q_{eq} = DT \]
    \[ s_{eq} = \frac{q_{eq}}{N} =  \frac{M}{MN +2} \]

    Prenons le cas où \(M=N\).

    Le salaire d'équilibre en concurrence pure et parfaite est alors \(\frac{1}{2}\).
    Le salaire d'équilibre en monopsone est de \(\frac{N}{N^{2} + 2}\).

    Quand le nombre d'individus est très grand, le salaire en concurrence pure
    et parfaite est toujours \(\frac{1}{2}\), alors qu'en monopsone, il tend vers 0 :
    l'entreprise préfère embaucher une proportion de plus en plus faible d'individus
    pour payer un salaire de plus en plus bas.

    \textbf{3. Salaire minimum}

    Analytiquement, on procède de la même manière que pour les questions précédentes :
    les individus maximisent leur utilité, les entreprises maximisent leur profit et
    l'on trouve le salaire et la quantité de travail d'équilibre en égalisant l'offre
    et la demande.

    La différence est que le salaire que les individus et entreprises considèrent
    est :
    \begin{itemize}
        \item Si le salaire d'équilibre en l'absence de salaire minimum aurait été
        inférieur à \(s_{min}\), le salaire minimum contraint les entreprises à payer
        au moins \(s_{min}\). Il faut donc résoudre le modèle pour la situation
        où \(s_{eq} = s_{min}\).
        \item Si le salaire d'équilibre en l'absence de salaire minimum aurait été
        supérieur à \(s_{min}\), le salaire minimum ne contraint  pas les entreprises à payer
        au moins \(s_{min}\) : elles le faisait déjà. Il faut donc résoudre le modèle pour la situation
        où le salaire d'équilibre est déterminé non par la contrainte, mais par l'offre et la
        demande.
    \end{itemize}

    \begin{figure}[p]
        \centering
        \includegraphics[width=10cm]{s_min_sup}
        \caption{Représentation graphique de l'effet d'un salaire minimum en
        concurrence pure et parfaite}
    \end{figure}

    En conséquence :
    \begin{enumerate}
        \item Si le salaire minimum est inférieur au salaire d'équilibre en concurrence pure
        et parfaite et au salaire d'équilibre en monopsone, il n'a aucun effet : les entreprises
        payaient déjà plus que le salaire minimum. Les forcer à payer davantage que \(s_{min}\)
        ne change donc rien.
        \item Si le salaire minimum est inférieur au salaire d'équilibre en concurrence pure
        et parfaite mais supérieur au salaire d'équilibre en monopsone, il n'a aucun effet en
        concurrence pure et parfaite, mais il force le monopsone à payer davantage ses employés,
        et donc à embaucher davantage.
        \item Si le salaire minimum est supérieur au salaire d'équilibre en concurrence pure et
        parfaite, l'offre de travail est supérieure à la demande.
    \end{enumerate}

	Le second point peut être démontré de la manière suivante. Le monopsone a deux choix : embaucher
	entre \(0\) et \(Ns_{min}\) unités de travail au salaire minimum,  ou embaucher une quantité
	supérieure à \(s_min\) et payer un salaire égal à \(q_{t} / N\).

	On est dans le cas où le salaire minimum est entre le salaire d'équilibre en monopsone et le salaire
	d'équilibre en concurrence pure et parfaite, c'est à dire :
	\[ \frac{M}{MN+2} \geq s_{min} \geq \frac{M}{M+N}\]

    Prenons un cas plausible du monopsone, par exemple \(M=1\) et \(N=100\)

	Dans le premier cas,
	\begin{align*}
	\frac{d\Pi}{dq} &= M - Mq - s_{min} \\
					&\geq M - MNs_{min} - s_{min} \\
					&\geq M - \left(MN+1\right) s_{min} \\
					&\geq M - \left(MN+1\right) s_{min} \\
					&\geq M - \left(MN+1\right) \frac{M}{M+N} \\
					&\geq M \left(1-\frac{MN+1}{M+N}\right) \\
					&\geq 1 \times \left( 1 - 1 \right) \\
					&\geq 0
	\end{align*}

	Si le monopsone embauche moins de \(Ns_{min}\) unités de travail, la dérivée de son
	profit est toujours positive ; autrement dit, le monopsone souhaite toujours
	augmenter le nombre d'heures de travail qu'il achète, jusqu'à atteindre le maximum
	de \(Ns_{min}\) unités de travail.

	Un calcul similaire montre que dans le second cas, le monopsone veut toujours
	diminuer sa consommation d'heures de travail jusqu'au minimum de \(Ns_{min}\) unités de travail.

    Dès lors, la quantité effectivement demandée par le monopsone est \(Ns_{min}\) et augmente
	donc lorsque \(s_{min}\) augmente.

	Dans le cas général, la quantité choisie par une entreprise sera atteinte quand la
	dérivée de son profit est égale à 0 :
	\begin{multline*}
	\frac{d\Pi}{dq} = \frac{d\left(q-\frac{q^{2}}{2} - sq \right)}{dq} = 0 \\
	\Rightarrow 1 - q^{*} - s'q^{*} - s = 0 \\
	\Rightarrow q^{*} = \frac{1-s}{1+s'}
	\end{multline*}

	En concurrence pure et parfaite, le salaire d'équilibre n'est pas affecté
	par la demande d'une seule entreprise : \(s' = 0\).

	En monopsone, \(s'>0\), et la quantité d'équilibre est donc plus faible en monopsone qu'en
	concurrence pure et parfaite.

	La déviation par rapport à l'équilibre compétitif sera d'autant plus importante
	que le salaire d'équilibre réagit à un changement de la quantité (ie.\ que \(s'\) est
	large), c'est à dire lorsque la demande est plus élastique.

    \textbf{4. Card \& Krueger}

    Les auteurs étudient l'augmentation du salaire minimum dans le New Jersey de
    \$4.25 à \$5.05 en 1992.

    Si le marché du travail était en concurrence pure et parfaite, l'effet
    attendu de cette mesure serait une baisse de l'emploi dans les entreprises
    affectées ; en monopsone ou oligopsone, l'effet attendu est une hausse de
    l'emploi et une répercussion sur les prix de vente des entreprises
    affectées.

    Les auteurs concluent que cette hausse du salaire minimum n'a pas eu d'effet
    négatif sur l'emploi ni d'effet positif sur les prix dans les entreprises
    les plus affectées.

\end{document}
