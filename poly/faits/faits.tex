\documentclass[../main.tex]{subfiles}

\begin{document}

    \section{}
    
    Ordres de grandreur (approximatifs) :
    \begin{itemize}
        \item \textsc{csg} : 100 milliards d'euros
        \item \textsc{ir} : 70 milliards d'euros
        \item \textsc{is} : 30 milliards d'euros
        \item \textsc{tva} : 150 milliards d'euros
        \item Cotisations sociales : 375 milliards d'euros
        \item Prélèvements obligatoires : 1000 milliards d'euros
        \item \textsc{Pib} : 2200 milliards d'euros
    \end{itemize}

    \section{Exercices}
   \textbf{4.} Quelle est la principale recette fiscale de de l'État ?
	    \begin{enumerate}
	    \item \textsc{ir}
	    \item \textsc{is}
	    \item \textsc{tva}
	    \item \textsc{ticpe}
	    \end{enumerate}

	\textbf{5.} Parmi ces trois catégories de prélèvements obligatoires, lesquelles
                sont les plus importantes (i.e.\ part plus importante du total) ?
		\begin{enumerate}
		\item Cotisations sociales \(>\) Impôts sur le revenu (dont \textsc{csg} et \textsc{is}) \(>\) \textsc{tva}
		\item Cotisations sociales \(>\) \textsc{tva} \(>\) Impôts sur le revenu (dont \textsc{csg} et \textsc{is})
		\item \textsc{tva} \(>\) Cotisations sociales \(>\) Impôts sur le revenu (dont \textsc{csg} et \textsc{is})
		\item \textsc{tva} \(>\) Impôts sur le revenu (dont \textsc{csg} et \textsc{is}) \(>\) Cotisations sociales
		\end{enumerate}

	\textbf{6.} Quel est le niveau de prélèvements obligatoires en France, en
    pourcentage du \textsc{pib} ?
		\begin{enumerate}
		\item 30 \%
		\item 45 \%
		\item 60 \%
        \item 75 \%
		\end{enumerate}
\end{document}
