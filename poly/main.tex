\documentclass[11pt, a4paper]{article}

\usepackage[french]{babel}

\usepackage[utf8]{inputenc}
\usepackage[T1]{fontenc}

\usepackage{xcolor}

\usepackage{amsmath}
\usepackage{amssymb}
\usepackage{amsfonts}

\usepackage{graphicx}

\usepackage{hyperref}
\hypersetup{
    colorlinks=true,
    linkcolor=red,
    }

\usepackage[]{geometry}
\setlength{\parindent}{0cm}
\setlength{\parskip}{.5\baselineskip}
\linespread{1.3}

\usepackage{mdframed}

\usepackage{natbib}
\bibliographystyle{unsrtnat}

\title{Économie publique}
\author{Benjamin Belrhomari$^\dagger$}
\date{$^\dagger$ \href{b@belrhomari.fr}{b@belrhomari.fr}}

\usepackage{subfiles} 

\begin{document}

    \maketitle

    \section{Introduction}

    Ce document rassemble une partie du matériel utilisé pour les travaux dirigés
    des cours d'économie publique des professeurs Hémet (2018) et Rojas (2019).
    Il n'est donc pas exhaustif.

    Les cours en question ont pour objectif de familiariser les étudiants avec
    la branche de l'économie s'intéressant au rôle de l'État. L'objectif du
    \textsc{td} est que les étudiants puissent lire de manière autonome des
    articles de recherche en économie (sans nécessairement être capable d'en
    dériver les résultats), d'en juger la qualité et d'en extraire les
    informations qui peuvent éclairer la décision publique.

    Les références suivantes sont conseillées :
    \begin{itemize}
        \item \textit{Économie des politiques publiques}, sous la direction de Julien
              Grenet et d'Antoine Bozio ;
        \item \dots
    \end{itemize}

    \section{Rappels de microéconomie}

        \subfile{./micro/microeconomie.tex}

    \section{Économie du bien-être}

    \section{Taxation}

    \section{Biens publics \& externalités}

    \section{Asymétries d'information \& assurances sociales}

    \section{Compétition et régulation des marchés}

    \section{Annexe}

        \subsection{Économie politique}

        \subsection{Économie comportementale}

        \subsection{Faits stylisés}

\end{document}


